\documentclass[a4paper]{article}

\usepackage[backend=biber,firstinits=true , backref=false, url=true, isbn=true]{biblatex}
\addbibresource{library.bib}

\usepackage{fullpage} % Package to use full page
\usepackage{amsmath}
\usepackage{hyperref}

\title{Playing Games: A Case Study in Active Learning Applied to Game Theory}
\author{Vince Knight}
\date{\today}

\begin{document}

\maketitle

\abstract{
A paper about active learning and using some example of this in a class on Game Theory
}

\section{Introduction}

Modern pedagogic theories as to how learning takes place such as constructivism
and socialism \cite{Illeris2009, Jordan2008a}, indicate that an \textbf{active
learning} approach is of benefit to student learning.  As stated in
\cite{Prince2004} there are a variety of complementary definitions of active
learning, however the general definition given in \cite{Prince2004} is the one
assumed in this paper:

\begin{quote}
``Active learning is generally defined as any instructional method that engages
students in the learning process. In short active learning requires students to
do meaningful learning activities and think about what they are doing.''
\end{quote}

One could argue that all learning is active as students simply listening to a
lecture are perhaps taking part in a `meaningful learning activity', however as
stated in \cite{Bonwell1991} active learning is understood to imply that
students:

\begin{itemize}
    \item read, write, discuss, or engage in solving problems;
    \item engage in higher order tasks such as analysis, synthesis and
        evaluation.
\end{itemize}

A variety of studies have highlighted the effectiveness of active learning
\cite{Freeman2014, Hake1998, Prince2004}. These two papers are in fact meta studies
evaluating the effectiveness an active student centred approach. Note that the
definition used in \cite{Freeman2014} corresponds to simply any pedagogic
approach in which students are not passive consumers of a lecture during the
class meeting.

Some examples of active learning in a variety of subjects include:

\begin{itemize}
    \item The flipped learning environment in a Physics class: \cite{Bates}.
    \item Inquiry based learning for the instruction of differential equations:
        \cite{Kwon2005}.
    \item Using collaborative learning in a pharmacology class:
        \cite{Depaz2008}.
\end{itemize}

The above sources (and references therein) generally discuss the pedagogic
approach from a macroscopic point of view with regards to the course considered.
This manuscript will give a detailed description of two particular active
learning activities used in the instruction of Game Theoretic concepts.

\section{An exemplar: a course in game theory}

\begin{itemize}
    \item Discuss other examples of how game theory is taught
    \item Describe this course
    \item Give description of the general philosophy of the course (no content:
        purely examples, which give data for games)
\end{itemize}

\subsection{Best responses}

\begin{itemize}
    \item The two thirds of the average game: how this is a basic game
    \item How it is played in class
    \item Description of data and discussion that ensues
\end{itemize}

\subsection{Repeated and random games}

\begin{itemize}
    \item The theory
    \item Tournaments:
        \begin{itemize}
            \item Basic type.
            \item Infinitely repeated game.
            \item Markov games.
        \end{itemize}
\end{itemize}

\section{Summary}

\begin{itemize}
    \item Give some examples of feedback.
    \item Mention how methods could be applied to other courses.
    \item Certain class management ideas (mainly that I will not speak first a
        lot of the time) <- Not sure if this is useful.
\end{itemize}

\printbibliography
\end{document}

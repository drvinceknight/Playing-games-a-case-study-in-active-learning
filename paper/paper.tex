\documentclass{article}

\usepackage[backend=biber,firstinits=true , backref=false, url=true, isbn=true]{biblatex}
\addbibresource{library.bib}

\usepackage{fullpage} % Package to use full page
\usepackage{amsmath}
\usepackage{hyperref}

\title{Playing Games: A Case Study in Active Learning Applied to Game Theory}
\author{Vince Knight}
\date{\today}

\begin{document}

\maketitle

\abstract{
A paper about active learning and using some example of this in a class on Game Theory
}

\section{Introduction}

Modern pedagogic theories as to how learning takes place such as constructivism
and socialism \cite{Illeris2009, Jordan2008a}, indicate that an \textbf{active
learning} approach is of benefit to student learning.  As stated in
\cite{Prince2004} there are a variety of complementary definitions of active
learning, however the general definition given in \cite{Prince2004} is the one
assumed in this paper:

\begin{quote}
``Active learning is generally defined as any instructional method that engages
students in the learning process. In short active learning requires students to
do meaningful learning activities and think about what they are doing.''
\end{quote}

One could argue that all learning is active as students simply listening to a
lecture are perhaps taking part in a `meaningful learning activity', however as
stated in \cite{Bonwell1991} active learning is understood to imply that
students:

\begin{itemize}
    \item read, write, discuss, or engage in solving problems;
    \item engage in higher order tasks such as analysis, synthesis and
        evaluation.
\end{itemize}

A variety of studies have highlighted the effectiveness of active learning
\cite{Freeman2014, Hake1998, Prince2004}. These two papers are in fact meta studies
evaluating the effectiveness an active student centred approach. Note that the
definition used in \cite{Freeman2014} corresponds to simply any pedagogic
approach in which students are not passive consumers of a lecture during the
class meeting. Some examples of active learning in a variety of subjects include:

\begin{itemize}
    \item The flipped learning environment in a Physics class: \cite{Bates}.
    \item Inquiry based learning for the instruction of differential equations:
        \cite{Kwon2005}.
    \item Using collaborative learning in a pharmacology class:
        \cite{Depaz2008}.
\end{itemize}

The above sources (and references therein) generally discuss the pedagogic
approach from a macroscopic point of view with regards to the course considered.
This manuscript will give a detailed description of two particular active
learning activities used in the instruction of Game Theoretic concepts:

\begin{itemize}
    \item Section \ref{sec:best_responses} will describe an in class activity
        and software package used to introduce students to the topic of best
        response dynamic \cite{Maschler2013}.
    \item Section \ref{sec:repeated_games} will describe an implementation of
        Axelrod's tournament \cite{Axelrod1980a,Axelrod1980a}.
\end{itemize}

These activities aim to introduce the student to the concepts and aspire to
their curiosity as to the underlying mathematics. Note that if there is any
doubt as to the effectiveness of active learning approaches, for example this paper
(the only one that this author could identify) \cite{Andrews2011} identifies no
such relationship are still beneficial to the students' learning.
Indeed in \cite{Poropat2014}  the greatest predictors of
academic performance are identified not as general intelligence \cite{Wright1905} but
personality factors such as conscientiousness and openness.

\section{An exemplar: a course in game theory}\label{sec:game_theory}

Game Theory as a topic is well suited to approaches that use activities
involving students as players to introduce the concepts, rules and strategies
for particular games and/or theorems presented.

In \cite{Brokaw2004} one such activity is presented: a game that allow players
to grasp the concept of common knowledge of rationality. Another good example is
\cite{Polak2008}: Yale's Professor Polak's course, the videos available at that
reference (a YouTube playlist) all show that students are introduced to every
concept through activity before discussing theory.

Just as the activity presented in \cite{Brokaw2004} the activities presented
here are both suited for as an early introduction to the concepts (although the
activity of Section \ref{sec:repeated_games} is potentially better suited to
being used at a later stage). Furthermore, these activities have also been used
as outreach activities for high school students with no knowledge of further
mathematics.

\subsection{Best response dynamics}\label{sec:best_responses}

The first step in this activity and potentially before any prior description of
Game Theory students are invited to answer the following simple question:

\begin{center}
    \textbf{What is a game?}
\end{center}

Through discussion the class will usually arrive at the following consensus:

\begin{itemize}
    \item A game must have a certain number \(N\geq 1\) of players;
    \item Each player must have available to them a certain number of strategies
        that define what they can do;
    \item Once all players have chosen their strategy, rules must specify what
        the outcome is.
\end{itemize}

This corresponds to the general definition of a strategic form game
\cite{Maschler2013}. The main goal of this activity is to not only understand
the vocabulary but also the important concept of response dynamics which aims
to identify what is the best option given prior knowledge of all other players
\cite{Maschler2013}. One particular game that can be analysed using base
response dynamics is often referred to:

\begin{center}
    \textbf{The two thirds of the average game.}
\end{center}

A good description of the game and the human dynamics associated to the play is
given in \cite{Nagel1995}. The rules are as follows:

\begin{itemize}
    \item All players choose a number between 0 and 100;
    \item The player whose choice was closest to \(\frac{2}{3}\) of the average
        of the choices wins.
\end{itemize}

The activity is similar to the activity described in
\cite{TheEconomicsNetwork2013}:

\begin{enumerate}
    \item Students are handed out copies of the form available at URL
    \item Students are explained the rules of the game and invited to play a
        first guess.
    \item After this the class is explained the rationalisation of the game
        (which uses best response dynamics to iteratively eliminate dominated
        strategies) as shown in Figure
        \ref{fig:rationalisation_of_two_thirds_game}.
    \item Students are invited to play one more time.
    \item Results are analysed and presented to the class for discussion.
\end{enumerate}

\subsection{Repeated and random games}\label{sec:repeated_games}

\begin{itemize}
    \item The theory
    \item Tournaments:
        \begin{itemize}
            \item Basic type.
            \item Infinitely repeated game.
            \item Markov games.
        \end{itemize}
\end{itemize}

\section{Summary}

\begin{itemize}
    \item Give some examples of feedback.
    \item Mention how methods could be applied to other courses.
    \item Certain class management ideas (mainly that I will not speak first a
        lot of the time) <- Not sure if this is useful.
\end{itemize}

\printbibliography
\end{document}
